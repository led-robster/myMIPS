\documentclass[a4paper,11pt]{article}
\usepackage[utf8]{inputenc}
\usepackage{geometry}
\usepackage{array}
\usepackage{booktabs}
\usepackage{xcolor}
\usepackage{longtable}
\usepackage[T1]{fontenc}
\usepackage{graphicx} % Required for \resizebox
\usepackage{adjustbox}
\geometry{margin=1in}

\pagecolor[RGB]{156,234,141} % Exact green background color from the image
\title{\textbf{myMIPS Instruction Set Architecture Documentation}}
\author{Led Robster}
\date{\today}

\begin{document}
	
	\maketitle
	
	\section*{Overview}
	myMIPS is an ISA that imitates the classic MIPS architecture. It is a byproduct of my interest in electronics and software and the result of my learning journey. May it be useful for at least one person other than me.
	
	\section*{Instruction Format}
	myMIPS is a 16-bit architetcure. Instructions are 16 bits wide, data is 16 bits wide. 
	Describe the instruction format, including fields such as opcode, operands, and immediate values. Use a table to illustrate this.
	
\begin{table}[h!]
	\centering
	\setlength{\arrayrulewidth}{0.8mm} % Thicker table lines
	\setlength{\tabcolsep}{3pt} % Reduced space between columns
	\renewcommand{\arraystretch}{1.4} % Increased row height
	\begin{tabular}{|c|c|c|c|c|c}
		\cline{1-5}
		\textbf{15--12} & \textbf{11--9} & \textbf{8--6} & \textbf{5--3} & \textbf{2--0} & \\ \cline{1-5}
		opcode & rs & rt & rd/shamt & Fcode & R-format\\ \cline{1-5}
		opcode & rs & rt & \multicolumn{2}{c|}{offset/immediate} & I-format \\ \cline{1-5}
		opcode & \multicolumn{4}{c|}{address} & J-format\\ \cline{1-5}
	\end{tabular}
	\caption{Instruction Format Example}
	\label{tab:instruction_format}
\end{table}
	
	Formal meaning. \\
	In R-format: rs = alu\_op1, rt = alu\_op2, rd = dest reg  \\
		\hspace*{2em} In R-format SLL or SRL : rs = des reg, rt = alu\_op1, shamt = alu\_op2  \\
	In I-format: rt = dest reg, rs = alu\_op1, imm = alu\_op2 ;  \\
		\hspace*{2em} In I-format BEQ : rs = alu op1, rt = alu op2
	
	
	\section*{Instruction Set}
	List all instructions, grouped by category (e.g., arithmetic, logical, memory, control). Include opcode, function codes, and descriptions.
	
	\subsection*{Arithmetic Instructions}
	Instructions involving arithmetic operators +, -.
	
	\resizebox{\textwidth}{!}{%
	\begin{tabular}{|l|l|l|l|p{6cm}|l|l|}
		\hline
		\textbf{Format} & \textbf{Instruction} & \textbf{Opcode} & \textbf{Fcode} & \textbf{Description} & \textbf{Example (CASM)} & \textbf{Effect}\\
		\hline
		R & ADD                  & 0000         & 000            & Add rs and rt, store in rd & ADD \$rd, \$rs, \$rt & rd=rs+rt\\
		\hline
		R & SUB                  & 0000         & 001            & Subtract rt from rs, store in rd & SUB \$rd, \$rs, \$rt & rd=rs-rt\\
		\hline
		I & ADDI                  & 0001         & ---            & Add rs and immediate, store in rt & ADDI \$rs, \$rt, d'10 & rt=rs+IMM\\
		\hline
	\end{tabular}%
	}

	
	\subsection*{Logical Instructions}
	Instructions involving logical operators \&, $|$, $<<$, $>>$. \\

	\resizebox{\textwidth}{!}{%
	\begin{tabular}{|l|l|l|l|p{6cm}|l|l|}
		\hline
		\textbf{Format} & \textbf{Instruction} & \textbf{Opcode} & \textbf{Fcode} & \textbf{Description} & \textbf{Example (CASM)} & \textbf{Effect}\\
		\hline
		R & AND                  & 0000         & 010            & Bitwise AND rs and rt, store in rd & AND \$rd, \$rs, \$rt & rd=rs AND rt\\
		\hline
		R & OR                   & 0000         & 011            & Bitwise OR rs and rt, store in rd & OR \$rd, \$rs, \$rt & rd=rs OR rt\\
		\hline
		R & SLL                  & 0000         & 101            & Shift-left-logical rt and shamt, store in rs & SLL \$rs, \$rt, shamt & rs=rt SLL shamt\\
		\hline
		R & SRL                  & 0000         & 110            & Shift-left-logical rt and shamt, store in rs & SRL \$rs, \$rt, shamt & rs=rt SRL shamt\\
		\hline
	\end{tabular}%
	}
	
	\subsection*{Comparison Instructions}
	Instructions involving relational operators $>$, ==. \\
	
	\resizebox{\textwidth}{!}{%
	\begin{tabular}[c]{|l|l|l|p{6cm}|l|l|}
		\hline
		\textbf{Instruction} & \textbf{Opcode} & \textbf{Fcode} & \textbf{Description} & \textbf{Example (CASM)} & \textbf{Effect} \\
		\hline
		SLT                  & 0000         & 100            & Set rd=1 if rs$<$rt & SLT \$rd, \$rs, \$rt & rs<rt ? rd=1 : rd=0\\
		\hline
		SLTI                   & 0011         & ---            & Set rt=1 if rs$<$IMM & SLTI \$rs, \$rt, IMM & rs<IMM ? rt=1 : rt=0\\
		\hline
	\end{tabular}%
	}
	
	
	\section*{Control Flow Instructions}
	List instructions used for branching and jumps, including encoding and behavior. \\
	
	\resizebox{\textwidth}{!}{%
	\begin{tabular}[c]{|l|l|l|p{6cm}|l|l|}
		\hline
		\textbf{Instruction} & \textbf{Opcode} & \textbf{Fcode} & \textbf{Description} & \textbf{Example (CASM)} & \textbf{Effect}\\
		\hline
		BEQ                  & 0110         & ---            & Branch to PC+OFFSET+1 if rs == rt  & BEQ \$rs, \$rt, OFFSET & rs==rt ? PC=PC+IMM+1\\
		\hline
		J                  & 0111         & ---            & Jump to ADDR & J h'F & PC=IMM\\
		\hline
		JAL                  & 1000         & ---              & Same as J but PC+1 is stored in \$ra  & JAL h'F & PC=IMM, ra=PC+1\\
		\hline
		JR                  & 0000         & ---              & Jump to address in register rs & JR \$rs & PC=rs\\
		\hline
	\end{tabular}%
	}
	
	\section*{Memory Instructions}
	Load/store instructions to access data memory. \\
	
	\resizebox{\textwidth}{!}{%
	\begin{tabular}[c]{|l|l|l|p{6cm}|l|l|}
		\hline
		\textbf{Instruction} & \textbf{Opcode} & \textbf{Fcode} & \textbf{Description} & \textbf{Example (CASM)} & \textbf{Effect}\\
		\hline
		LW                   & 0100         & ---            & Load RAM[rs+OFFSET] in rt & LW \$rs, \$rt, OFFSET & rt=RAM[rs+IMM]\\
		\hline
		SW                   & 0101         & ---            & Store rt in RAM[rs+offset] & SW \$rs, \$rt, OFFSET & RAM[rs+IMM]=rt\\
		\hline
	\end{tabular}%
	}
	
	\section*{Register Set}
	Describe the registers available in your ISA, including their names, sizes, and special purposes.
	
	\begin{table}[h!]
		\centering
		\begin{tabular}{|c|c|p{8cm}|}
			\hline
			\textbf{Register Name} & \textbf{Size (bits)} & \textbf{Purpose/Description} \\
			\hline
			r0                    & 32                   & Always zero \& CONTROL reg\\
			\hline
			r1-r7                 & 32                   & General Purpose\\
			\hline
			r8                    & 32                   & Stack pointer \\
			\hline
			r14                   & 32                   & Return address \\
			\hline
		\end{tabular}
		\caption{Register Set}
		\label{tab:register_set}
	\end{table}
	
		\section*{Encoding Examples}
	Provide encoding examples for a few representative instructions.
	
	\begin{table}[h!]
		\centering
		\begin{tabular}{|l|l|l|l|l|}
			\hline
			\textbf{Instruction} & \textbf{Opcode} & \textbf{rs} & \textbf{rt} & \textbf{Effect}\\
			\hline
			ADD \$rd, \$rs, \$rt       & 0110011         & 00010         & 00011         & rd=rs + rt \\
			\hline
			SUB \$rd, \$rs, \$rt       & TBD         & TBD         & TBD          & rd=rs - rt\\
			\hline
			ADDI \$rs, \$rt, d'10       & TBD         & TBD         & TBD          & rt=rs + 10 \\
			\hline
			AND \$rd, \$rs, \$rt       & TBD         & TBD         & TBD          & rd=rs AND rt \\
			\hline
			OR \$rd, \$rs, \$rt       & TBD         & TBD         & TBD          & rd=rs OR rt \\
			\hline
			ANDI \$rs, \$rt, h'0A       & TBD         & TBD         & TBD          & rt=rs AND 0x0A \\
			\hline
			SLL \$rs, \$rt, d'2       & TBD         & TBD         & TBD          & rs=rt sll 2 \\
			\hline
			SRL \$rs, \$rt, d'2       & TBD         & TBD         & TBD          & rs=rt srl 2 \\
			\hline
			LW \$rs, \$rt, d'11       & TBD         & TBD         & TBD          & rt=RAM[rs+11] \\
			\hline
			SW \$rs, \$rt, d'11       & TBD         & TBD         & TBD          & RAM[rs+11]=rt \\
			\hline
			BEQ \$rs, \$rt, d'16       & TBD         & TBD         & TBD          & rs==rt ? PC=PC+16+1 : PC=PC+1 \\
			\hline
			J d'11       & TBD         & TBD         & TBD          & PC=11 \\
			\hline
			JR \$rs       & TBD         & TBD         & TBD          & PC=rs \\
			\hline
			JAL d'7       & TBD         & TBD         & TBD          & PC=7 , ra=PC+1 \\
			\hline
			SLT \$rd, \$rs, \$rt       & TBD         & TBD         & TBD          & rs<rt ? rd=1 : rd=0 \\
			\hline
			SLTI \$rs, \$rt, h'0A       & TBD         & TBD         & TBD          & rs<0x0A ? rt=1 : rt=0 \\
			\hline
			
		
			
		\end{tabular}
		\caption{Encoding Examples}
		\label{tab:encoding_examples}
	\end{table}
	
	\section*{Additional Notes}
	
	\textbf{CASM Immediate} : immediates can be written in the following format b'XXX, h'YYY, d'ZZZ. Depends on the number base preferred. b' is for binary format, h' for hexadecimal format and 'd for decimal format. Immediates are 6-bits long, writing anything higher than 2**6-1 will result in truncation. \\
	Examples: b'111; d'10; h'3F
	
	\textbf{REGISTER ZERO} : MIPS arch includes a zero-register. Reading at zero-reg returns 0. Writing at zero-reg has no effect. This last sentence is not valid in myMIPS. Writing to zero-reg has the effect of writing to CONTROL-reg. CONTROL-reg handles minor configurations for the CPU.
	
	\vspace{0.3cm}
	
	\noindent \textbf{REGISTER 0: \hspace{0.3cm} CONTROL REGISTER}
	
	\renewcommand{\arraystretch}{1.3}
	\setlength{\tabcolsep}{8pt}
	\begin{tabular}{|c|c|c|c|c|c|c|c|}
		\hline
		R/W-0 & R/W-0 & R/W-0 & R-1 & R-1 & R/W-x & R/W-x & R/W-x \\
		\hline
		IRP & RP1 & RP0 & TO & PD & Z & DC & C \\
		\hline
		bit15 &  &  &  &  &  &  & bit0 \\
		\hline
	\end{tabular}
	
	\vspace{0.3cm}
	
	\begin{description}
		\item[bit 15-1] \textbf{Unimplemented:} Maintain as ‘0’.
		
		
		\item[bit 0] \textbf{C:} Regfile bank bit \\
		\hspace*{0.5cm} 1 = BANK1, opens access to r9-r15 \\
		\hspace*{0.5cm} 0 = BANK0, opens access to r1-r7
	\end{description}
	
	\vspace{0.2cm}
	
	\noindent \textbf{Note:} A subtraction is executed by adding the two’s complement of the second operand. \\
	For rotate (\texttt{RRF, RLF}) instructions, this bit is loaded with either the high or low order bit of the source register.
	
	\vspace{0.5cm}
	
	\noindent
	\begin{tabular}{|c|l|}
		\hline
		R & Readable bit \\
		W & Writable bit \\
		U & Unimplemented bit, read as ‘0’ \\
		n & Value at POR \\
		‘1’ & Bit is set \\
		‘0’ & Bit is cleared \\
		x & Bit is unknown \\
		\hline
	\end{tabular}
	
	
\end{document}
